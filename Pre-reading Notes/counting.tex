\section{Counting}
\begin{example}
    Suppose a restaurant has 5 types of cake and 2 types of ice cream to select for dessert.
    \begin{enumerate}
        \item How many choices for dessert are there if you select one cake and one ice cream?
        There are $5\cdot 2 = 10$ choices. \emph{This is a sequence of tasks - first, choose a dessert, then, choose an ice cream}.
        \item How many choices for dessert are there if you select either one cake or one ice cream, but not both?
        There are $5 + 2 = 7$ choices. \emph{These are two distinct cases}.
    \end{enumerate}
\end{example}

\begin{example}
    Consider a password consisting of 3 letters from the set $\left\{A,\,B,\,C,\,\dots,\,Z\right\}$.
    \begin{enumerate}
        \item How many passwords are possible?
        There are 26 choices for each of the 1st, 2nd and 3rd letters, so there are $$26\cdot 26\cdot 26 = 17576 \text{ possiblities}$$
        \item How many passwords contain no repeated letters?
        There are 26 letters for the 1st letter, 25 for the 2nd and 24 for the 3rd.
        $$\therefore 26\cdot 25\cdot 24 = 15600 \text{ possiblities}$$
        \item How many passwords use only vowels or only consonants?
        $$\text{option } 1 \text{ (only vowels) } + \text{option } 2\text{ (only consonants)}$$
        $$5^3 + 21^3 = 9386$$
    \end{enumerate}
\end{example}

\begin{definition}
    A \underline{permutation} of a set of objects is an arrangement of the objects into an order.
\end{definition}

\begin{example}
    How many permutations of the letters in the word \emph{SWITCH} are there? i.e. SWITCH, CWITHS, etc.

    $$\text{\# of these} = 6\cdot 5\cdot 4\cdot 3\cdot 2\cdot 1 = 6! = 720$$
\end{example}

\begin{theorm}
    For any $n\in\mathbb Z^+$, the number of permutations of a set with $n$ elements is $n!$.
\end{theorm}

\begin{example}
    Consider the permutations of the letters of the word \emph{OBJECTS}. How many permutations start with a vowel?

    Option 1: start with ``o'': $6\cdot 5\cdot 4\cdot 3\cdot 2\cdot 1 = 6!$ \\
    Option 2: start with ``e'': $6\cdot 5\cdot 4\cdot 3\cdot 2\cdot 1 = 6!$
    \\
    Answer: $6!+6! = 1440$.
\end{example}

\begin{example}
    Consider all license plates consisting of 3 digits from the set $\left\{0,\,1,\,\dots,\,9\right\}$ followed by 3 letters from the set $\left\{A,\,B,\,\dots,\,Z\right\}$

    \begin{enumerate}
        \item How many license plates are possible?
        $$10\cdot 10\cdot 10 \cdot 26\cdot 26\cdot 26 = 17576000$$
        \item How many license plates have no repeated symbols?
        $$10\cdot 9\cdot 8\cdot 26\cdot 25\cdot 24 = 11232000$$
        \item How many license plates have at least one repeated symbol?
        We know the total number of license plates and the number of those with no repeated symbols, so the rest must have at least one repeated symbol. $$\therefore 17576000 - 11232000 = 6344000$$
    \end{enumerate}
\end{example}

Let $n$ and $r$ to be non-negative integers.

Problem: \emph{Select $r$ elements from a set containing $n$ elements. How many ways are there to do the selection?}

The answer depends on:
\begin{itemize}
    \item whether or not order matters
    \item whether or not repetition is allowed
\end{itemize}

Case where order matters and repetition is allowed.

\begin{example}
    We have a new ATM card and need to select a PIN. We may choose four digits from the set 
    \[\{0,\,1,\,2,\,3,\,4,\,5,\,6,\,7,\,8,\,9\}\]
    Order matters, and we may repeat. How many PINs are there?

    In each of the four selections we have 10 choices, and hence there are $10^4$ possibilities.
\end{example}

Fact: the number of selections of $r$ elements from a set containing $n$ elements where order matters and repetition is allowed is $n^r$.

Case where order matters, and there is no repetition.

\begin{example}
    If there are 7 runners, how many ways can 1st, 2nd and 3rd be awarded? $$7\cdot 6\cdot 5 = 210$$
\end{example}

\begin{definition}
    Let $n$ and $r$ be non-negative integers with $r\leq n$. An \underline{r-permutation} of a set of $n$ elements is an ordered selection of $r$ elements taken from the set of $n$ elements. The number of r-permutations of a set of $n$ elements is denoted $P(n,r)$ or $nPr$.
\end{definition}

\newpage
\begin{theorm}
    If $n,\,r\in\mathbb Z$ and $1\leq r\leq n$, then $$P(n,\,r) = n(n-1)(n-2)\dots (n-r+1) = \frac{n!}{\left(n-r\right)!}$$
\end{theorm}

Case where order doesn't matter, and repetition is not allowed.

\begin{example}
    In how many ways can $5$ students be selected from a class of $15$ to form a committee?

    If we assume order did matter, we would count as in case 2, to get $$15\cdot 14\cdot 13\cdot 12\cdot 11 = 360360$$

    But order does not matter, and each set of students was counted $5! = 120$ times, so the actual number of choices is $$\frac{360360}{120} = 3003$$
\end{example}

\begin{definition}
    Let $n$ and $r$ be non-negative integers with $r\leq n$. An \underline{r-combination} of a set of $n$ elements is a subset of $r$ of the $n$ elements. The number of $r$-combinations of a set of $n$ elements is denotes $C(n,r)$ or $nCr$ or, more commonly, $\left(^n_r\right)$ which is read ``$n$ choose $r$''.
\end{definition}

\begin{theorm}
    If $n$ and $r$ are non-negative integers and $r\leq n$, then $$\left(^n_r\right) = \frac{P(n,r)}{r!} = \frac{n!}{r!(n-r)!}$$
\end{theorm}

Case where order does not matter, and repetition is allowed.

\begin{example}
    Suppose a store has 4 large buckets, each with a different type of coloured candy: red, blue, yellow, pink. If you must select a total of $7$ candies, how many different choices do you have?

    Select $7$ elements from $\{r,\,b,\,y,\,p\}$ (repetition allowed) where order doesn't matter (i.e. ``rrbbyyy'' is the same as ``ryrybby'')

    Answer = $120$.
\end{example}

\newpage
\begin{example}
    How many permutations of the letters ``ALFALFA'' are there?

    Method (1).
    \begin{itemize}
        \item select positions for the 3 A's
        \item select positions for the 2 F's
        \item put the 2 L's in the remaining positions
    \end{itemize}

    Answer: $$\left(^7_3\right) \cdot \left(^4_2\right) \cdot 1 = 35 \cdot 6 = 210$$ note the $1$ is really $\left(^2_2\right) = 1$


    Method (2). if the letters were distinct, $$A_1 A_2 A_3 F_1 F_2 L_1 L_2$$ there would be $7! = 5040$ possiblities. Now, make the 3 A's indistinguishable. We have counted $$A_1 F_1 F_2 L_2 L_1 A_2 A_3$$ and $$A_2 F_1 F_2 L_2 L_1 A_1 A_3$$ as different solutions, so we have over counted by $3! = 6$ times. Thus, $\frac{5040}{6} = 840$ possible solutions.

    Now make the 2 F's indistinguishable: we have over counted by $2$ times, so we have $\frac{840}{2!} = 420$ possible solutions. Finally, make the 2 L's indistinguishable: we have over counted by $2!$ times, so we have $\frac{420}{2!} = 210$ possible solutions.
\end{example}

\begin{theorm}
    Suppose you have $n$ objects of which $n_1$ are if type $1$, $n_2$ are of type $2$, etc. The number of distinct permutations of the $n$ objects is $$\left(^n_{n_1}\right)\left(^{n-n_1}_{n_2}\right)\left(^{n-n_1-n_2}_{n)3}\right)\dots\left(^{n-n_1-n_2-\dots -n_{k-1}}_{n_k}\right) = \frac{n!}{n_1!n_2!\dots n_k!}$$
\end{theorm}

\subsection{Inclusion and Exclusion}
\begin{example}
    In a fishtank, there are $14$ blue fish, $7$ striped fish, and $4$ fish that are both blue and striped. How many fish are blue or striped?

    If we add the blue fish and the striped fish, we get $14+7=21$. However, we counted the blue striped fish twice (once for being blue, once for being striped), so the actual answer is $(14+7)-4=17$.
\end{example}

\begin{theorm}
    Let $A$ and $B$ be disjoint finite sets, the $|A\cup B| = |A| + |B|$
\end{theorm}

More generally,
\newpage
\begin{theorm}
    For any finite sets $A$ and $B$, $|A\cup B| = |A|+|B| - |A\cap B|$
\end{theorm}

\begin{definition}
    The \underline{Inclusion/Exclusion Principle} (for $2$ or $3$ sets)...

    If $A$, $B$, $C$ are any finite sets, then
    $$|A\cup B| = |A| + |B| - |A\cap B|$$
    and
    $$|A\cup B\cup C| = |A| + |B| + |C| - |A\cap B| - |A\cap C| - |B\cap C| + |A\cap B\cap C|$$
\end{definition}

\subsection{Pigeonhole Principle}
\begin{definition}
    \underline{The Pigeonhole Principle}: Suppose you have $n$ pigeons sitting in $k$ pigeonholes. If $n > k$, then at least one of the pigeonholes contains at least two pigeons.

    \begin{example}
        5 pigeons, 4 holes, at least 1 hole must have more than 1 pigeon.
    \end{example}

    Equivalently, a function from a finite set to a smaller finite set cannot be one-to-one.
\end{definition}

The \underline{contrapositive} of the pigeonhole principle is: Suppose you have $n$ pigeons sitting in $k$ pigeonholes. If each pigeonhole contains at most one pigeon, then $n\leq k$.

\begin{example}
    If you have $3$ different colours in your drawer, what is the minimum number of socks you need to pull out in order to guarantee a matching pair?

    Think of pigeons = socks, pigeonholes = colours. Pull out $4$ socks to guarantee that at least two of them have the same colour.
\end{example}

\begin{example}
    There are $680$ people in a list. Must there be at least two people on the list with the same first and last initials? Explain.

    pigeons = people, pigeonholes = initials.

    There are $26\cdot 26 = 676$ possible options for first and last initials. Since $680 > 676$, the pigeonhole principle implies at least 2 people must have the same initials.
\end{example}
\newpage
\begin{definition}
    \underline{The generalised pigeonhole principle}

    Suppose you have $n$ pigeons sitting in $k$ pigeonholes. If $n>k\cdot m$, then at least one of the pigeonholes contains at least $m+1$ pigeons.

    Contrapositive: suppose you have $n$ pigeons sitting in $k$ pigeonholes. If each pigeonhole contains at most $m$ pigeons, then $n\leq km$.
\end{definition}

\begin{example}
    Show that in a group of $25$ people, at least $3$ must be born in the same month.

    Let $n=25$ and $m=2$.

    We have $25$ pigeons (people) and $12$ pigeonholes (months).

    Since $n > 12\cdot 2$, the generalised pigeonhole principle implies that there is a month which $\geq 3$ people from the group have a birthday.
\end{example}
