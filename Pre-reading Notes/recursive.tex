\section{Recursive Defintions}
\begin{example}
    The sequence $$1,\,1,\,2,\,3,\,5,\,8,\,13,\,21,\,34,\,55,\,\dots$$ is called the \underline{Fibonacci Sequence}
\end{example}

\begin{definition}
    A \underline{recurrence relation} for a sequence $a_0,a_1,a_2,\dots$ is a formula that relates each term $a_k$ to some of its predecessors $a_{k-1},\dots,a_{k-i}$ where $i\in\mathbb Z$ and $k-i \geq 0$

    The \underline{initial conditions} for such a recurrence relation specify the values of some of the intial terms.
\end{definition}

\begin{example}
The Fibonacci sequence is defined recursively by $$F_0 = 1,\, F_1 = 1,\,\text{and}\,F_n = F_{n-1} + F_{n-2}$$
\end{example}

\subsection{Ways to Define Sequences}
A sequence can be defined
\begin{itemize}
    \item \emph{informally}, by listing the first few terms of the sequence until the pattern becomes obvious
    \item \emph{with a general formula}, by stating how a term $a_n$ depends on $n$ and stating where it starts
    \item \emph{recursively}, by giving a recurrence relation relating terms in the sequence to earlier ones and also some intial conditions
\end{itemize}

\subsection{Showing a Sequence Satisfies a Recurrence Relation}
\begin{example}
Show that the sequence $$a_k = 3\cdot 2^k,\,\text{for }k\geq 0$$ satisfies the recurrence relation $$a_n = 2a_{n-1},\,\text{for }n\geq 1$$

The sequence is $\displaystyle \left\{3\cdot 2^k\right\}_{k\geq 0} = 3,\,6,\,12,\,24,\,48,\,dots$

For every integer $n\geq 1$ we have $a_n = 3\cdot 2^n$ and $a_{n-1} = 3\cdot 2^{n-1}$

Hence, $$a_n = 3\cdot 2^n = 3\cdot 2\cdot 2^{n-1} = 2 (3\cdot 2^{n-1}) = 2 a_{n-1}$$
\end{example}

\subsection{Generalised}
We have seen that sequences of numbers can be defined recursively. Many other objects can be defined recursively as well, such as: sets, sums, products and function.

A recursive definition for a set of objects requires three things:
\begin{enumerate}
    \item BASE: a statement that a certain object belongs in the set
    \item RECURSION: a collection of rules showing how to form new objects for the set from existing ones in the set
    \item RESTRICTION: a statement that no objects belong to the set other than those arising from steps 1 and 2
\end{enumerate}

\begin{example}
    Consider the set of all \underline{valid bracketings}. Every left bracket ( is matched with a right bracket ) and at every stage, reading left to right, there are at least as many left brackets as right brackets.
    $$(())()\text{ is valid}$$
    $$()()()\text{ is valid}$$
    $$())(()\text{ is invalid}$$

    \underline{Recursive definition of the set of valid brakcetings}
    \begin{enumerate}
        \item Base: an empty expression with no brackets is valid
        \item Recursion: \begin{enumerate}
            \item if B is valid, the (B) is also valid
            \item if B and C are valid, then BC is also valid
        \end{enumerate}
        \item Restriction: Any expression not derived from the rules above is not valid
    \end{enumerate}
\end{example}

\begin{definition}
    Arithmatic Sequences are defined as such:
    \[a_k = a_{k-1} + d \qquad a_n = a_0 +dn\]
    Geometric Sequences are defined as such:
    \[a_k = r\cdot a_{k-1} \qquad a_n = a_0\cdot r^n\]
\end{definition}

