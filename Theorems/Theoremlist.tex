\documentclass[12pt]{article}
\usepackage{graphicx}
\usepackage{float} \restylefloat{table}
\usepackage{amssymb}
\usepackage{amsmath}
\usepackage{amsthm}
\usepackage{multicol}
\usepackage{arydshln}
\usepackage{tikz}
\usepackage{xcolor}
\usepackage{wrapfig}
\usepackage[top=2cm,bottom=2cm,left=2cm,right=2cm]{geometry}
\usepackage{enumerate}
\usepackage{todonotes}
\usepackage{cancel}
\usepackage[framemethod=TikZ]{mdframed}
\usepackage{mathtools}

\definecolor{peach}{RGB}{255,218,185}

% customise settings
\setcounter{tocdepth}{2}

\numberwithin{figure}{section}
\numberwithin{table}{section}

\setlength{\parindent}{0pt}

\newcommand{\deriv}{\,\mathrm{d}}


% \mdfsetup{skipabove=\topskip,skipbelow=\topskip}

% THEOREM
\definecolor{purp}{RGB}{255,0,255}
\newcounter{theo}[section]\setcounter{theo}{0}
%\renewcommand{\thetheo}{\arabic{section}.\arabic{theo}}
\newenvironment{theo}[2][]{%
%\refstepcounter{theo}%
\ifstrempty{#1}%
{\mdfsetup{%
frametitle={%
\tikz[baseline=(current bounding box.east),outer sep=0pt]
\node[anchor=east,rectangle,fill=purp!40]
{\strut Theorem~\thetheo};}}
}%
{\mdfsetup{%
frametitle={%
\tikz[baseline=(current bounding box.east),outer sep=0pt]
\node[anchor=east,rectangle,fill=purp!40]
{\strut Theorem: ~#1};}}%
}%
\mdfsetup{innertopmargin=10pt,linecolor=purp!40,%
linewidth=2pt,topline=true,%
frametitleaboveskip=\dimexpr-\ht\strutbox\relax
}
\begin{mdframed}[]\relax%
\label{#2}}{\end{mdframed}}
% To use this, \begin{theo}[bruh theorem]{}



% DEFINITION
\newcounter{defi}[section] \setcounter{defi}{0}
\renewcommand{\thedefi}{\arabic{section}.\arabic{defi}}
\newenvironment{definition}[2][]{%
\refstepcounter{defi}%
\ifstrempty{#1}%
{\mdfsetup{%
frametitle={%
\tikz[baseline=(current bounding box.east),outer sep=0pt]
\node[anchor=east,rectangle,fill=yellow!40]
{\strut Definition~\thedefi};}}
}%
{\mdfsetup{%
frametitle={%
\tikz[baseline=(current bounding box.east),outer sep=0pt]
\node[anchor=east,rectangle,fill=yellow!40]
{\strut Definition:~#1};}}%
}%
\mdfsetup{innertopmargin=10pt,linecolor=yellow!40,%
linewidth=2pt,topline=true,%
frametitleaboveskip=\dimexpr-\ht\strutbox\relax
}
\begin{mdframed}[]\relax%
\label{#2}}{\end{mdframed}}



% LEMMMA
\newcounter{lem}[section]\setcounter{lem}{0}
\renewcommand{\thelem}{\arabic{section}.\arabic{lem}}
\newenvironment{lem}[2][]{%
\refstepcounter{lem}%
\ifstrempty{#1}%
{\mdfsetup{%
frametitle={%
\tikz[baseline=(current bounding box.east),outer sep=0pt]
\node[anchor=east,rectangle,fill=blue!40]
{\strut Lemma~\thelem};}}
}%
{\mdfsetup{%
frametitle={%
\tikz[baseline=(current bounding box.east),outer sep=0pt]
\node[anchor=east,rectangle,fill=blue!40]
{\strut Lemma:~#1};}}%
}%
\mdfsetup{innertopmargin=10pt,linecolor=blue!40,%
linewidth=2pt,topline=true,%
frametitleaboveskip=\dimexpr-\ht\strutbox\relax
}
\begin{mdframed}[]\relax%
\label{#2}}{\end{mdframed}}
% To use this, \begin{lem}[bernouli ineq]{}


% EXAMPLE
\newcounter{exam}[section] \setcounter{exam}{0}
\renewcommand{\theexam}{\arabic{section}.\arabic{exam}}
\newenvironment{example}[2][]{%
\refstepcounter{exam}%
\ifstrempty{#1}%
{\mdfsetup{%
frametitle={%
\tikz[baseline=(current bounding box.east),outer sep=0pt]
\node[anchor=east,rectangle,fill=orange!40]
{\strut Example~\theexam};}}
}%
{\mdfsetup{%
frametitle={%
\tikz[baseline=(current bounding box.east),outer sep=0pt]
\node[anchor=east,rectangle,fill=orange!40]
{\strut Example:~#1};}}%
}%
\mdfsetup{innertopmargin=10pt,linecolor=orange!40,%
linewidth=2pt,topline=true,%
frametitleaboveskip=\dimexpr-\ht\strutbox\relax
}
\begin{mdframed}[]\relax%
\label{#2}}{\end{mdframed}}

% PROPOSITION
\newcounter{prop}[section] \setcounter{prop}{0}
\renewcommand{\theprop}{\arabic{section}.\arabic{prop}}
\newenvironment{proposition}[2][]{%
\refstepcounter{prop}%
\ifstrempty{#1}%
{\mdfsetup{%
frametitle={%
\tikz[baseline=(current bounding box.east),outer sep=0pt]
\node[anchor=east,rectangle,fill=green!40]
{\strut Proposition~\theprop};}}
}%
{\mdfsetup{%
frametitle={%
\tikz[baseline=(current bounding box.east),outer sep=0pt]
\node[anchor=east,rectangle,fill=green!40]
{\strut Proposition:~#1};}}%
}%
\mdfsetup{innertopmargin=10pt,linecolor=green!40,%
linewidth=2pt,topline=true,%
frametitleaboveskip=\dimexpr-\ht\strutbox\relax
}
\begin{mdframed}[]\relax%
\label{#2}}{\end{mdframed}}

% NOTE
\newenvironment{note}%
{\begin{quote}\textbf{Note}:}%
{\end{quote}}


% OBSERVATION
\newenvironment{observation}%
{\begin{quote}\textbf{Obs}:}%
{\end{quote}}

% TOOL
\newenvironment{tool}%
{\begin{quote}\textbf{Tool}:}%
{\end{quote}}

% FACT
\newenvironment{fact}%
{\begin{quote}\textbf{Fact}:}%
{\end{quote}}

% IDEA
\newenvironment{idea}%
{\begin{quote}\textbf{Idea}:}%
{\end{quote}}

% Corollary
\newenvironment{corollary}%
{\begin{quote}\textbf{Corollary}:}%
{\end{quote}}

% Question
\newenvironment{question}%
{\begin{quote}\textbf{Q}:}%
{\end{quote}}

% Remark
\newenvironment{remark}%
{\begin{quote}\textbf{Rmk}:}%
{\end{quote}}

% Notation
\newenvironment{notation}%
{\begin{quote}\textbf{Notation}:}%
{\end{quote}}


%quality of life

\newcommand{\Lim}{\displaystyle\lim}
\newcommand{\integral}{\displaystyle\int}
\sloppy
\title{1071 Final Theorem List}
\author{Ben Kruger}

%Proof

\newenvironment{prf}[1][]{%
\ifstrempty{#1}%
{\mdfsetup{%
frametitle={%
\tikz[baseline=(current bounding box.east),outer sep=0pt]
\node[anchor=east,rectangle,fill=red!50]
{\strut Proof};}}
}%
{\mdfsetup{%
frametitle={%
\tikz[baseline=(current bounding box.east),outer sep=0pt]
\node[anchor=east,rectangle,fill=red!50]
{\strut Proof:~#1};}}%
}%
\mdfsetup{innertopmargin=10pt,linecolor=red!50,%
linewidth=2pt,topline=true,%
frametitleaboveskip=\dimexpr-\ht\strutbox\relax
}
\begin{mdframed}[]\relax%
}{\qed\end{mdframed}}

% To use this, \begin{prf}{}

    % DERIVATION
\newcounter{deri}[section] \setcounter{deri}{0}
\renewcommand{\thederi}{\arabic{section}.\arabic{deri}}
\newenvironment{derivation}[2][]{%
\refstepcounter{deri}%
\ifstrempty{#1}%
{\mdfsetup{%
frametitle={%
\tikz[baseline=(current bounding box.east),outer sep=0pt]
\node[anchor=east,rectangle,fill=orange!40]
{\strut Derivation~\thederi};}}
}%
{\mdfsetup{%
frametitle={%
\tikz[baseline=(current bounding box.east),outer sep=0pt]
\node[anchor=east,rectangle,fill=orange!40]
{\strut Derivation:~#1};}}%
}%
\mdfsetup{innertopmargin=10pt,linecolor=orange!40,%
linewidth=2pt,topline=true,%
frametitleaboveskip=\dimexpr-\ht\strutbox\relax
}
\begin{mdframed}[]\relax%
\label{#2}}{\end{mdframed}}
\allowdisplaybreaks
\usepackage{hyperref}
\begin{document}
\maketitle
\newpage
\tableofcontents
\newpage
%This is the start of the prf%
\section{Squeeze Theorem}
\begin{theo}[Squeeze Theorem For Sequences]{}
    Suppose \(\left(a_n\right)^\infty _{n=1}\), \(\left(b_n\right)^\infty _{n=1}\), \(\left(c_n\right)^\infty _{n=1}\) are such that
    \begin{enumerate}
        \item \[a_n \leq b_n \leq c_n \qquad \forall n \in \mathbb{N}\]

        \item \[\Lim_{n\to\infty} a_n = \Lim_{n\to\infty} c_n = L\]
    \end{enumerate}
    Then,
    \[\Lim_{n\to\infty} b_n = L\]
\end{theo}
\begin{prf}{}
    Observe that,  
\begin{align*}
|b_n - L| &= |b_n - a_n + a_n - L|\\
|(b_n - a_n) + (a_n - L)| &\le |b_n - a_n| + |a_n - L| = b_n - a_n + |a_n - L|\\
&\le c_n - a_n + |a_n - L| = |c_n - L + L - a_n| + |a_n - L|\\
&\le |c_n - L| + |L - a_n| + |a_n - L|\\
\end{align*}
Fix $\varepsilon > 0$. There exists $N_1 \in \mathbb N$ such that if $n \ge N_1$ then $$|a_n - L| = |L - a_n| < \frac{\varepsilon}{3}$$.\\
Also, there is $N_2 \in \mathbb N$ such that if $n \ge N_2$ then $$|c_n - L| < \frac{\varepsilon}{3}$$
Now, set $N = \max{N_1, N_2}.$ If $n \ge N$ then, $$|b_n - L| \le |c_n - L| + |L - a_n| + |a_n - L| < \frac{\varepsilon}{3} + \frac{\varepsilon}{3} + \frac{\varepsilon}{3} = \varepsilon$$ Thus, $\Lim_{n\to\infty}{b_n} = L$. \\
\end{prf}
\newpage
\begin{explanation}{}
    The squeeze theorem takes two sequences either side of the sequence in question that converge to the same limit. This means that the sequence in question is squeezed towards the same limit as one sequence is larger for all elements and the other is smaller.\\    
    The proof starts by taking the limit definition of \(b_n\) and then expands it by adding a clever 0. The triangle inequality is pretty clutch and makes this look nice. We have \[b_n - a_n + |a_n - L|\]
    We can replace the \(b_n\) with \(c_n\) because it is larger.\\
    Now it is a matter of stating the result of the definition for the two limits we assumed. Now we can get the exact inequality we want with epsilon and \(b_n\).
\end{explanation}



\section{A Convergent Sequence is Bounded}
\begin{theo}[A Convergent Sequence is Bounded]{}
    Convergent sequences are bounded. More precicely, if $(a_n)_{n=1}^\infty$ converges, then there exists $M > 0$ such that $|a_n| \le M$ for all $n \in \mathbb N$\\
\end{theo}
\begin{prf}{}
    Assume $\Lim_{n\to\infty}{a_n} = L$.\\
For every $\varepsilon > 0$, there exists $N \in \mathbb N$ such that if $n \ge N$, then $|a_n - L| < \varepsilon$.\\
This holds for instance if $\varepsilon = 1$.\\
Thus, there exists $N_1 \in \mathbb N$ such that if $n \ge N_1$, then $$|a_n - L| < 1$$\\
By the second triangle inequality, $$|a_n - L| \ge |a_n| - |L|$$\\
Thus, $|a_n| - |L| < 1$ and $|a_n| < 1 + |L|$ \\
Now, $|a_n| \le M = \max\{|a_1|, |a_2|, |a_3|, \cdots, |a_{n-1}|, |L| + 1\}$ for all $n \in \mathbb N$.
\end{prf}
\newpage
\begin{explanation}{}
This theorem describes how if a sequence converges to a point, it must not go off to infinity at any point, that is it is bounded by some arbitrary M both positively and negatively.\\
The proof rearranges the limit definition to bound the sequence. Taking the max ensures that the bound is defined as the largest term of the sequence + 1.\\
\end{explanation}



\section{A Bounded, Monotone Sequence Converges}
\begin{theo}[A Bounded, Monotone Sequence Converges]{}
    A monotone sequence converges if and only if it is bounded.
\end{theo}
\begin{prf}{}
    $\left(\implies\right)$\\ This is already proven, see 2\\
$\left(\impliedby\right)$\\ Without loss of generality, assume $(a_n)_{n=1}^\infty$ is increasing (If the sequence is monotone decreasing, a similar argument is used).\\ Define $\alpha = \sup\{a_1, a_2, a_3, \cdots, a_n\} \in \mathbb R$.\\ Since the sequence is bounded, this supremum exists in $\mathbb R$. \\
Let us prove $$\alpha = \Lim_{n\to\infty}{a_n}$$ \\ Chose $\varepsilon > 0$. \\By an earlier theorem, there exists $N \in \mathbb N$ such that $a_N \in (\alpha - \varepsilon, \alpha]$.\\ By monotonicity, if $n\ge N$ then $a_n \ge a_N > \alpha - \varepsilon$. Also, $a_n \le \alpha$. \\ Thus, if $n \ge N$ then $\alpha - \varepsilon < a_n \le \alpha < \alpha + \varepsilon \implies |a_n - \alpha| < \varepsilon$. Thus, $\Lim_{n\to \infty}{a_n} = \alpha$, so $(a_n)_{n=1}^\infty$ converges.\\
\end{prf}
\begin{explanation}{}
    This theorem continues from the previous theorem by proving the other direction, if a sequence is bounded, it must converge. That means, if there is some point that the sequence cannot go above, then the sequence must eventually tend to a point. This is ensured by assuming the sequence is monotone as then the sequence will only get larger or smaller to the bound.\\
    The part that uses an 'earlier theorem' simply says that for some large enough N, we have the element is delta away from the limit. Monotonicity implies that each succesive term is larger than the last (or smaller).
\end{explanation}



\section{Existence of a Monotone Subsequnce}
\begin{theo}[Existence of a Monotone Subsequnce]{}
    Every sequence of real numbers has a monotone subsequence.
\end{theo}
\begin{prf}{}
We call $k \in \mathbb N$ a peak point of $(a_n)_{n=1}^\infty$ if $a_k > a_n$ for all $n > k$. \\
\underline{Case 1:} There are infinitely many peak points, $n_1 < n_2, n_3 < \cdots$. By definition, $a_{n_1} > a_{n_2} > a_{n_3} >  \cdots$. Thus, $(a_{n_k})_{k=1}^\infty$ is monotone.\\
\underline{Case 2:} There are finitely many peak points. $k_1, k_2, k_3, \cdots, k_l$. Set $$n_1 = \max\{k_1, \cdots, k_l\} + 1$$ Clearly, $n_1$ is not peak. Therefore, there is some $n_2 > n_1$ such that $a_{n_2} \ge a_{n_1}$. Now, $n_2$ is not peak. Therefore there is $n_3 > n_2$ such that $a_{n_3} \ge a_{n_2}$. Continue in this way to obtain a monotone subsequence $(a_{n_i})_{i=1}^\infty$.
\end{prf}
\begin{explanation}{}
    This theorem states that each sequence has a monotone subsequence. It is proven using the idea of peak points. If we have infinite peak points, then the subsequence will only be able to decrease by the definition of a peak point.\\
    The second case examines if there are finite peak points. Here, the argument that is used is that by definition of a peak point, we can say that there are elements bigger than those bigger than the peak points. This creates some monotone increasing sequence.     
\end{explanation}



\section{Bolzano-Weierstrass Theorem}
\begin{theo}[Bolzano-Weierstrass Theorem]{}
    Every bounded sequence has a convergent subsequence.
\end{theo}
\begin{prf}{}
    Every sequence has a monotone subsequence. Every subsequence of a bounded sequence is bounded, thus, we have a bounded, monotone subsequence which must converge. 
\end{prf}
\begin{explanation}{}
    If you don't get this, get out of this course... Honestly, for your own good.
\end{explanation}



\section{Limit Arithmetic}
\begin{theo}[Limit Arithmetic]{}
Assume \(\Lim_{x\to a}f(x) = L \quad \Lim_{x\to a}g(x) = m\).\\
Then,
\begin{enumerate}
    \item \(\Lim_{x\to a}(f(x) + g(x)) = L + m\)
    \item \(\Lim_{x\to a}(f(x)\cdot g(x)) = L\cdot m\)
    \item \(\Lim_{x\to a}\frac{f(x)}{g(x)} = \frac{L}{m}\)\\
\end{enumerate}    
\end{theo}
\begin{prf}{}
\begin{enumerate}
    \item Fix \(\varepsilon > 0\). We need to find \(\delta > 0\) such that if \(0 < |x - a| < \delta\) then \(|f(x)+ g(x) - L + m| < \varepsilon\).\\
    \begin{align*}
        |f(x) + g(x) - (L + m)| &= |f(x) + g(x)  - (L + m)|\\
        &= |(f(x) - L) + (g(x) - m)|\\
        &\le |f(x) - L| + |g(x) - m|\\
    \end{align*}
    Since \(\Lim_{x\to a}f(x) = L, \quad \Lim_{x\to a}g(x) = m\), there exists \(\delta_1, \delta_2 > 0\) such that if \(0 < |x - a| < \delta_1, \delta_2\) then, 
    \begin{align*}
        |f(x) - L| &< \frac{\varepsilon}{2}\\
        |g(x) - m| &< \frac{\varepsilon}{2}\\
    \end{align*}
    Choosing \(\delta = \min\{\delta_1, \delta_2\}\), we have that,
    \begin{align*}
    |f(x) - L| + |g(x) - m|  &< \frac{\varepsilon}{2} + \frac{\varepsilon}{2}\\
    &= \varepsilon
    \end{align*}
    \item Fix \(\varepsilon > 0\). We need to find \(\delta > 0\) such that if \(0 < |x - a| < \delta\) then \(|f(x)g(x) - Lm| < \varepsilon\).\\
    \begin{align*}
        |f(x)g(x) - Lm| &= |f(x)g(x) - f(x)m + f(x)m - Lm|\\
        &= |f(x)(g(x) - m) + m(f(x) - L)|\\
        &\le |f(x)||g(x) - m| + |m||f(x) - L|\\
    \end{align*}
    Since \(\Lim_{x\to a}f(x) = L\), there exists \(\delta > 0\) such that if \(0 < |x - a| < \delta\) then, \(|f(x) - L| < 1\).\\
    In this case, \(|f(x)| - |L| < 1 \implies |f(x)| < 1 + |L|\).\\
    Thus, if \(0 < |x - a| < \delta\), then \(|f(x)g(x) - Lm| \le |1 + |L|||g(x) - m| + |m||f(x) - L|\).\\
    There exists \(\delta_2 > 0\) such that if \(0 < |x-a| < \delta_2\) then, \(|g(x) - m| < \frac{\varepsilon}{2(|m| + 1)}\).\\
    Set \(\delta = \min\{\delta_1, \delta_2, \delta_3\}\),\\
    If \(0 < |x-a| < \delta\) then,
    \begin{align*}
        |f(x)g(x) - Lm| &\le (1+|L|)|g(x) - m| + |m||f(x) - L|\\
        &< \frac{\varepsilon}{2(|L| + 1)}\cdot(|L| + 1) + \frac{\varepsilon}{2(|m| + 1)} \cdot 2(|m| + 1)
        &\le \frac{\varepsilon}{2} + \frac{\varepsilon}{2}\\
        &= \varepsilon 
    \end{align*}
    \item We need to show first that \(\Lim_{x\to a}\frac{1}{g(x)} = \frac{1}{m}\)\\
    Fix \(\varepsilon > 0\). \\
    Since \(\Lim_{x\to a}g(x) = m\), there exists \(\delta_1 > 0\) such that if \(0 < |x - a| < \delta_1\) then, \(|g(x) - m| < \frac{|m|}{2}\).\\
    \begin{align*}
        |m| &= |m + g(x) - g(x)|\\
        &\le |g(x) - m| + |g(x)|\\
        &< \frac{|m|}{2} + |g(x)|\\
        \frac{|m|}{2} &< |g(x)|\\
        \frac{1}{g(x)} &< \frac{2}{|m|}
    \end{align*}
    Indeed, there exists some \(\delta_2 > 0\) such that if \(0 < |x - a| < \delta_2\) then \(|g(x) - m| < \frac{|m|^2}{2}\varepsilon\).\\
    Choosing \(\delta = \min\{\delta_1, \delta_2\}\), if \(0 < |x - a| < \delta\) we have,\\
    \begin{align*}
        \bigg\vert\frac{1}{g(x) - \frac{1}{m}}\bigg\vert &= \bigg\vert\frac{m - g(x)}{mg(x)}\bigg\vert\\
        &= \frac{1}{|mg(x)|}|g(x) - m|\\
        &< \frac{1}{|m|}\frac{2}{|m|}|g(x) - L|\\
        &< \frac{2}{|m|^2}\frac{|m|^2}{2}\varepsilon\\
        &= \varepsilon
    \end{align*}
    Now we have proven that \(\Lim_{x\to a} \frac{1}{g(x)} = \frac{1}{m}\), the more general fact follows.\\
    \begin{align*}
        \Lim_{x\to a} \frac{f(x)}{g(x)} &=  \Lim_{x\to a} f(x) \frac{1}{g(x)}\\
        &= \Lim_{x\to a} f(x)\Lim_{x\to a}\frac{1}{g(x)}\\
        &= L\frac{1}{m}\\
        &= \frac{L}{m}
    \end{align*}
\end{enumerate}   
\end{prf}
\begin{explanation}{}
    This theorem basically describes the basic arithmetic that works with limits. The proof of parts 1 and 2 are trivial, they come directly from the definition of the limit. Part 3 works by proving a specific case of $\frac{1}{g(x)}$ then applying the multiplication proven in 2.
\end{explanation}




\section{Differentiability implies Continuity}
\begin{proposition}[Differentiability implies Continuity]{}
    Suppose $f$ is differentiable at $x_0$. Then $f$ is continuous at $x_0$.  
\end{proposition}
\newpage
\begin{prf}{}
Observe that:\\
\begin{align*}
    \lim_{x\to\ x_0}{f(x)} &= \lim_{x\to x_0}{\left(\frac{f(x) - f(x_0)}{x - x_0}\cdot(x-x_0) + f(x_0)      \right)}\\ 
    &= \lim_{x\to x_0}{\frac{f(x) - f(x_0)}{x - x_0}} \cdot \lim_{x\to x_0}{(x - x_0)} + f(x_0)\\
    &= f'(x_0)\cdot (x_0 - x_0) + f(x_0)\\
    &= 0\cdot f'(x_0) + f(x_0) = f(x_0)
\end{align*}
Therefore, $f$ is continuous at $x_0$.
\end{prf}
\begin{explanation}{}
    This theorem says that a function being differentiable at a point will ensure its continuous. This is because of the fact that for a function to be differentiable, it needs to approach some limit from the left and right. Those being the same will imply a derivative and indeed will imply continuity.\\
    The proof adds a clever 0 to give the definition of the derivative.
\end{explanation}


\section{Derivative Arithmetic}
\begin{theo}[Derivative Arithmetic]{}
Suppose \(f,g: (a, b) \to \mathbb R\) are both differentiable at \(x \in (a, b)\). Then, \(f + g, fg, \frac{f}{g}\) are differentiable at \(x\). We need \(g(x) \neq 0\) for \(\frac{f}{g}\).\\
\begin{enumerate}
    \item \((f+g)'(x) = f'(x) + g'(x)\)
    \item \((fg)'(x) = f'(x)g(x) + g'(x)f(x)\)
    \item \(\left(\frac{f}{g}\right)'(x) = \frac{f'(x)g(x) - g'(x)f(x)}{g^2(x)}\) \\
\end{enumerate}    
\end{theo}
\newpage
\begin{prf}{}
\begin{enumerate}
    \item 
    \begin{align*}
    (f+g)'(x) &= \Lim_{h\to 0}\frac{f(x+h) + g(x + h) - f(x) - g(x)}{h}\\
    &= \Lim_{h\to 0}\frac{f(x+h) - f(x) + g(x+h) - g(x)}{h}\\
    &= \Lim_{h\to 0}\frac{f(x+h)-f(x)}{h} + \Lim_{h\to 0}\frac{g(x+h) - g(x)}{h}\\
    &= f'(x) + g'(x)   
    \end{align*}
    \item 
    \begin{align*}
    (fg)'(x) &= \Lim_{h\to 0}\frac{f(x+g)g(x+h) - f(x)g(x)}{h}\\
    &= \Lim_{h\to 0}\frac{f(x+h)g(x+h) - f(x+h)g(x) + f(x+h)g(x) - f(x)g(x)}{h}\\
    &= \Lim_{h\to 0}f(x+h)\frac{g(x+h) - g(x)}{h} + g(x)\frac{f(x+h) - f(x)}{h}\\
    &= \Lim_{h\to 0}f(x+h)\Lim_{h\to 0}\frac{g(x+h) - g(x)}{h} + g(x)\Lim_{h\to 0}\frac{f(x+h) - f(x)}{h}\\
    &= f(x)g'(x) + f'(x)g(x)
    \end{align*}
    \item 
    \begin{align*}
    \left(\frac{f'(x)}{g'(x)}\right) &= \Lim_{h\to 0}\frac{\frac{f(x+h)}{g(x+h)} - \frac{f(x)}{g(x)}}{h}\\
    &=  \Lim_{h\to 0}\frac{\frac{f(x+h)}{g(x+h)}\cdot\frac{g(x)}{g(x)} - \frac{f(x)}{g(x)}\cdot\frac{g(x+h)}{g(x+h)}}{h}\\
    &= \Lim_{h\to 0}\frac{g(x)f(x+h) - f(x)(g(x+h))}{hg(x)g(x+h)}\\
    &= \Lim_{h\to 0}\frac{g(x)f(x+h) -f(x)g(x) + f(x)g(x) - f(x)(g(x+h))}{hg(x)g(x+h)}\\
    &= \Lim_{h\to 0}\frac{g(x)(f(x+h) - f(x)) - f(x)(g(x) - g(x+h))}{hg(x)g(x+h)}\\  
    &= \frac{g(x)\Lim_{h\to 0}\frac{f(x+h) - f(x)}{h} - f(x)\Lim_{h\to 0}\frac{g(x+h) - g(x)}{h}}{\Lim_{h\to 0}g(x)g(x+h)}\\
    &= \frac{g(x)f'(x) - f(x)g'(x)}{g^2(x)}
    \end{align*}
\end{enumerate}    
\end{prf}
\begin{explanation}{}
    Much like the limit arithmetic theorem, this theorem lists and proves certain properties of derivatives.\\
    Again, part 1 and 2 are very straight forward and follow directly from the definition of the derivative.\\
    Part 3 has a bit of algebruh envolved but is fairly similar.
\end{explanation}


\section{Rolle's Theorem}
\begin{theo}[Rolle's Theorem]{}
    Suppose $f: [a, b] \rightarrow \mathbb R$ is continuous on $[a, b]$ and differentiable on (a, b). If $f(a) = f(b)$, then there exists $c \in (a, b)$ such that $f'(c) = 0$.\\
\end{theo}
\begin{prf}{}
    If $f(x) = f(a)$ for all $x \in (a, b)$, then the theorem is obvious. \\Assume $f$ is not constant.\\ Without loss of generality, assume there exists $x_0 \in (a, b)$ such that $f(x_0) > f(a)$. \\(If no such $x_0$ exists, then there exists $x_1 \in (a, b)$ such that $f(x_1) < f(a)$. In this case, a similar argument works.)\\
    By the Extreme Value theorem, $f$ has a global maximum on $[a, b]$, call this $c$. Since,\\ $$f(c) \ge f(x_0) > f(a) = f(b)$$ we know $c \neq a, c \neq b$. Thus, $c$ lies in $(a, b)$, and $f'(c) = 0$
\end{prf}
\begin{explanation}{}
    Rolle's Theorem takes some function that is differentiable and continuous. It places a condition on it such that the end points are equal. If this is so, there is a point of derivative 0 between them. This is because if the end points are equal, either the function is constant or it changes direction at some point. This change in direction will have a derivative of 0.\\
    The proof outlines the initial trivial case of a constant function. It then assumes that we have some value that is greater than the end points (this does not lose generality as a similar argument follows for value less than the end point). The extreme value theorem predicts some global maximum in the interval. This is larger than the assumed point and is thus larger than the end points meaning there is a derivative 0 within the interval.
\end{explanation}




\newpage
\section{The Mean Value Theorem}
\begin{theo}[The Mean Value Theorem]{}
    Suppose $f: [a, b] \rightarrow \mathbb R$ is continuous on $[a, b]$ and differentiable on $(a, b)$. Then, there exists $c \in (a, b)$ such that $f'(c) = \frac{f(b) - f(a)}{b- a}$.
\end{theo}
\begin{prf}{}
    Consider the function $\phi(x) = f(x) - f(a) - \frac{f(b) - f(a)}{b - a}\cdot (x - a)$.\\ Observe that\\
\begin{align*}
    \phi(a) &= f(a) - f(a) - \frac{f(b) - f(a)}{b - a}\cdot (a - a) = 0\\
    \phi(b) &= f(b) - f(a) - \frac{f(b) - f(a)}{\cancel{b - a}}\cdot \cancel{(b - a)} = 0
\end{align*}
Applying Rolle's theorem to $\phi(x)$, we obtain the existence of $c \in (a, b)$ such that, \\
  $$\phi'(c) = f'(c) - \frac{f(b) - f(a)}{b - a} = 0$$  
\end{prf}
\begin{explanation}{}
    This theorem is similar to Rolle's Theorem. It says that there is some derivative of a function at a point that will be equal to the average gradient between the end points.\\
    The proof is completed by considering a function that when Rolle's Theorem is applied to will give exactly what is wanted.\\
\end{explanation}




\section{A Vanishing Derivative Implies a Constant Function}
\begin{proposition}[Vanishing Derivative Implies a Constant Function]{}
    Suppose $f: [a, b] \implies \mathbb R$ is continuous on $[a, b]$ and differentiable on $(a, b)$. If $f'(x) = 0$ for all $x \in (a, b)$, then $f$ is constant on $[a, b]$\\
\end{proposition}
\newpage
\begin{prf}{}
    Take $x \in (a, b]$. Applying the Mean Value theorem on $[a, x]$, we conclude that $f(x) - f(a) = f'(c)(x - a)$ for some $c \in (a, x)$. Therefore, as $f'(c) = 0$, \\
\begin{align*}
    f(x) - f(a) &= 0(x - a)\\
    f(x) &= f(a)
\end{align*}
Thus, $f$ is constant in $[a, b]$.
\end{prf}
\begin{explanation}{}
    This is a specific case of the mean value theorem. We use the fact that the mean value theorem gives and set the derivative to be 0. This gives us a constant function.
\end{explanation}



\section{A Continuous Function on an interval is Uniformly Continuous}
\begin{theo}[A Continuous Function on an interval is Uniformly Continuous]{}
Suppose \(f\) is continuous on a closed, bounded interval \([a, b]\). Then, \(f\) is uniformly continuous on \([a, b]\).    
\end{theo}
\begin{prf}{}
By contradiction.\\
Assume f is not uniformly continuous.\\ There exists \(\varepsilon_0 > 0\) such that for all \(\delta > 0\), there exists \(x,y \in [a, b]\) with \(|x - y| < \delta\) but \(|f(x) - f(y)| \ge \varepsilon_0\).\\
Take \(\delta = 1\). There exists \(x_1, y_1 \in [a, b]\) such that \(|x_1 - y_1| < 1\) but \(|f(x_1) - f(y_1)| \ge \varepsilon_0\).\\
Take \(\delta = \frac{1}{2}\). There exists \(x_2, y_2 \in [a, b]\) such that \(|x_2 - y_2| < \frac{1}{2}\) but \(|f(x_2) - f(y_2)| \ge \varepsilon_0\).\\
Continue in this way for \(\delta = \frac{1}{n}, \quad \forall n \in \mathbb N\).\\
Then, there exists \(x_n, y_n \in [a, b]\) such that \(|x_n - y_n| < \frac{1}{n}\) but \(|f(x_n) - f(y_n)|\ge \varepsilon_0\).\\
We have thus constructed two sequences.\\
\[\left(X_n\right)_{n=1}^\infty\]
\[\left(Y_n\right)_{n=1}^\infty\]
inside \([a, b]\).\\
Since \([a, b]\) is closed and bounded, \(\left(X_n\right)_{n=1}^\infty\) must have some subsequence \(\left(X_{n_k}\right)_{k=1}^\infty\) which converges to some point \(x_0 \in [a, b]\).
\[\Lim_{k\to\infty}\left(X_{n_k}\right) = x_0\]
\underline{Claim:}\\
The subsequence \(\left(Y_{n_k}\right)_{k=1}^\infty\) converges to \(x_0\).\\
\underline{Proof of the claim:}\\
Given \(\varepsilon > 0\), we look for \(N \in \mathbb N\) such that if \(k \ge N\) then, \(|Y_{n_k} - x_0| < \varepsilon\).\\
\begin{align*}
    |Y_{n_k} - x_0| &= |Y_{n_k} + X_{n_k} - X_{n_k} - x_0|\\
    &\le |Y_{n_k} - X_{n_k}| + |X_{n_k} - x_0|\\
    &< \frac{1}{n_k} + |X_{n_k} - x_0|
\end{align*}
Since \(\left(X_n\right)_{n=1}^\infty\) converges to \(x_0\), for k large enough, we have \(|X_{n_k} - x_0| < \frac{\varepsilon}{2}\).\\
Then taking k large enough such that \(\frac{1}{n_k} < \frac{\varepsilon}{2}\), we have, 
\begin{align*}
    |Y_{n_k} - x_0| &< \frac{1}{n_k} + |X_{n_k} - x_0|\\
    &< \frac{\varepsilon}{2} + \frac{\varepsilon}{2}\\
    &= \varepsilon
\end{align*}
Let us prove that this contradicts the continuity of f.\\
Since f is continuous and since \[\Lim_{k\to\infty}X_{n_k} = \Lim_{n\to\infty}Y_{n_k} = x_0\]
we have that, 
\[\Lim_{k\to\infty}f(X_{n_k}) = \Lim_{n\to\infty}f(Y_{n_k}) = f(x_0)\]
Then, for some \(N \in \mathbb N\), if \(k \ge N\) then \(|f(X_{n_k}) - x_0| < \frac{\varepsilon}{4}\) and \(|f(Y_{n_k}) - x_0| < \frac{\varepsilon}{4}\).\\
On the other hand, 
\begin{align*}
    |f(X_{n_k}) - f(Y_{n_k})| &= |f(X_{n_k}) - f(x_0)  + f(x_0) - f(Y_{n_k})|\\
    &\le |f(X_{n_k}) - f(x_0)| + |f(x_0) - f(Y_{n_k})|\\
    &< \frac{\varepsilon}{4} + \frac{\varepsilon}{4}\\
    &= \frac{\varepsilon}{2}
\end{align*}
Also, we know by construction that \[|f(X_{n_k}) - f(Y_{n_k})| \ge \varepsilon_0\]
Thus, we have a contradiction.
\end{prf}
\begin{explanation}{}
    This theorem states that any function on a closed bounded interval will be uniformly continuous if its continuous on that interval.\\
    The proof starts by constructing two sequences using the definition of a limit multiple times for decreasing delta values. We then can say that since the interval is bounded, there is a convergent subsequence. We then prove that both subsequences converge to the same limit. This is a simple use of the definition. \\
    We then apply the definition of the limit to the sequences if they were put into the function. This gives us a contradiction to the initial assumption that it was not uniformly continuous and we are done.\\
\end{explanation}


\section{Integrability Condition}
\begin{theo}[Integrability Condition]{}
The function \(f:[a, b] \to \mathbb R\) is integrable if and only if for every \(\varepsilon > 0\), there exists a partition \(P\) such that \[U(f, P) - L(f, P) < \varepsilon\]
\end{theo}
\begin{prf}{}
\(\left(\implies\right)\)\\
Assume f is integrable. Fix \(\varepsilon > 0\). There exists \(P_1\) such that \[\integral_a^b f(x)\deriv x = \integral_{\underline{a}}^bf\deriv x < L(f, P_1) + \frac{\varepsilon}{2}\]
Also, there exists \(P_2\) such that 
\[\integral_a^b f(x)\deriv x = \integral_{a}^{\overline{b}}f\deriv x > U(f, P_1) - \frac{\varepsilon}{2}\]
Define \(P = P_1 \cup P_2\). Then,
\begin{align*}
    U(f, P) - L(f, P) &\le U(f, P_2) - L(f, P_1)\\
    &< \integral_a^b f(x) \deriv x + \frac{\varepsilon}{2} - \integral_a^b f(x)\deriv x + \frac{\varepsilon}{2}\\
    &= \varepsilon
\end{align*}
\(\left(\impliedby\right)\)\\
Fix \(\varepsilon > 0\). There exists \(P\) such that,
\[U(f, P) - L(f, P) < \varepsilon\]
Then, \[\integral_a^{\overline{b}}f(x)\deriv x - \integral_{\underline{a}}^bf(x)\deriv x \le U(f, P) - L(f, P) <  \varepsilon\]
This means, \(\integral_a^{\overline{b}}f(x)\deriv x - \integral_{\underline{a}}^bf(x)\deriv x = 0\).\\
Thus, f is integrable.
\end{prf}
\begin{explanation}{}
    This is almost considered the definition of Riemann Integration. It states that a function is integrable if and only if we can create some partitions such that the distance between the lower and upper sums is small.\\
    In the forward direction, we know f is integrable. We start with some integral and a partition to describe the lower sum. We do the same for the upper sum. Defining a partition as the sum of these two allows us to create an inequality using the assumed partitions and this new union.\\
    The reverse direction gives us the distance between the partitions and we must show that we have an integrable function. We create an inequality using the upper and lower integrals of f. We then can see that the upper and lower integral are the same which means the function is integrable.\\
\end{explanation}


\section{Finite Discontinuities implies Integrability}
\begin{theo}[Finite Discontinuities implies Integrability]{}
If \(f:[a, b]\to \mathbb R\) is continuous at all but finitely many points, then \(f\) is integrable.    
\end{theo}
\begin{prf}{}
\underline{Part 1:}\\
Assume f is continuous on \([a, b]\) then f is uniformly continuous on \([a, b]\).\\
Fix \(\varepsilon > 0\). We will find a partition \(P\) such that \(U(f, P) - L(f, P) < \varepsilon\). This will imply integrability.\\
Uniform continuity implies that there exists \(\delta > 0\) such that if \(|x - y| < \delta\) then \(|f(x) - f(y)| < \frac{\varepsilon}{|b-a|}\).\\
Choose \(P\) so that \(|x_i - x_{i-1}| < \delta \) for all \(i\in 1, 2, \cdots , n\).\\
Here \(P = \{x_0, x_1, x_2, \cdots, x_{n-1}\}\). Then, \[U(f, P) - L(f, P) = \displaystyle\sum_{i=1}^n\left(\displaystyle\sup_{[x_{i-1}, x_i]}f(x) - \displaystyle\inf_{[x_{i-1}, x_i]}f(x)\right)\left(x_i - x_{i-1}\right)\]
Since f is continuous, by the extreme value theorem, there exists \(x_i' \in [x_{i-1}, x_{i}]\) such that, \[f(x_i') = \displaystyle\sup_{[x_{i-1}, x_i]}f(x)\]
And there exists some \(x_i'' \in [x_{i-1}, x_{i}]\) such that, \[f(x_i'') = \displaystyle\inf_{[x_{i-1}, x_i]}f(x)\]
Then, 
\begin{align*}
    U(f, P) - L(f, P) &= \displaystyle\sum_{i=1}^n\left(f(x_i') - f(x_i'')\right)\left(x_i - x_{i-1}\right)\\
    &< \displaystyle\sum_{i=1}^n \frac{\varepsilon}{|b - a|} \left(x_i - x_{i-1}\right)\\
    &= \frac{\varepsilon}{|b - a|} \left(x_1 - x_0 + x_2 - x_1 + \cdots + x_n - x_{n-1}\right)\\
    &=  \frac{\varepsilon}{|b - a|}\left(x_n - x_0\right)\\
    &= \frac{\varepsilon}{|b - a|}|b - a|\\
    &= \varepsilon
\end{align*}
Thus, f is integrable.\\
\underline{Part 2:}\\
Assume f has exactly 1 discontinuity, assume it is at \(C \in (a, b)\). The cases where it is at a or b are treated similarly.\\
Fix \(\varepsilon > 0\). We will find a partition \(P\) such that \(U(f, P) - L(f, P) < \varepsilon\).\\
Define \(\delta = \min\{\frac{\varepsilon}{8M}, \frac{b - C}{2}, \frac{C - a}{2}\}\).\\
Where M is such that \(|f(x)| \le M \quad \forall x \in [a, b] \).\\
By our choice of \(\delta, [C-\delta, C+\delta] \subset [a, b]\).\\
By part 1, f is integrable on \([a, C-\delta]\) and \([C+\delta, b]\).\\
Therefore, there exist partitions \(P_1\) of \([a, C-\delta]\) and \(P_2\) of \([C+\delta, b]\) such that,
\begin{align*}
    U(f, P_1) - L(f, P_1) &< \frac{\varepsilon}{4}\\
    U(f, P_2) - L(f, P_2) &< \frac{\varepsilon}{4}
\end{align*}
Define \(P = P_1 \cup P_2\), a partition of \([a, b]\).\\
\begin{align*}
    U(f, P) - L(f, P) &= U(f, P_1) - L(f, P_1) + U(f, P_2) - L(f, P_2) +\\ &\left(\displaystyle\sup_{[C-\delta, C+\delta]}f(x) - \displaystyle\inf_{[C+\delta, C-\delta]}\right)f(x)(C+\delta - (C-\delta))\\
    &< \frac{\varepsilon}{4} + \frac{\varepsilon}{4} + 2M\cdot 2\delta\\
    &\le \frac{\varepsilon}{2} + \frac{4\varepsilon}{8M}\cdot M\\
    &= \frac{\varepsilon}{2} + \frac{\varepsilon}{2}\\
    &= \varepsilon
\end{align*}
\underline{Part 3:}\\
If f has more than 1 discontinuity, apply part 2 enough times.
\end{prf}
\begin{explanation}{}
This theorem describes how a continuous function at all but finitely many points is still integrable. The idea behind this is that if we can take an area around each discontinuity od width epsilon, make the width very small and then as long as it isnt infinitely tall, the area will be negligible, thus making it still integrable.\\
Part 1 of the proof looks at a function with no discontinuities. This is relatively simple. We know the function is uniformly continuous so we use that definition and we need to take a partition such that the upper and lower sum is less than epsilon. We expand this into the summation notation and use extreme value theorem to predict two extremes, the sup and inf. This is then rearranged to show the partition is less than epsilon.\\
The second part of the proof looks at 1 discontinuity. We assume that the discontinuity is at a point C and then set delta to be the minimum of some values. We know that the function is bounded. By the choice of delta, no matter what, there is some distance around C within the interval. By part 1, we have that its integegrable on either side of C. We then make those the partitions. Defining a third partition as the union of the two halfs, we get the entire interval minus a bit around C. We can define the third partition by adding the other two and adding the middle section around C. This is then rearranged to give less than epsilon which shows integrability. \\Part 3 simply applies part 2 multiple times.
\end{explanation}


\section{A Monotone Function is Integrable}
\begin{theo}[A Monotone Function is Integrable]{}
If \(f:[a, b] \to \mathbb R\) is monotone, then \(f\) is integrable.    
\end{theo}
\begin{prf}{}
Assume without loss of generality that \(f\) is increasing.\\
Fix \(\varepsilon > 0\). We will find a partition \(P = \{x_0, x_1, x_2, \cdots, x_n\}\) such that \(U(f, P) - L(f, P) < \varepsilon\).\\
Let \(P\) be the partition that splits \([a, b]\) into n equal parts.\\
Namely, set \(x_k = a+ k\cdot \frac{b-a}{n}\). Then,
\begin{align*}
    U(f, P) - L(f, P) &= \displaystyle\sum_{i=1}^n\left(\left(\displaystyle\sup_{[x_{i-1}, x_i]}f(x) - \displaystyle\inf_{[x_{i-1}, x_i]}f(x)\right)\left(x_{i} - x_{i-1}\right)\right)\\
    &= \displaystyle\sum_{i=1}^n\left(f(x_i)-f(x_{i-1})\right)\left(\frac{b - a}{n}\right)\\
    &= \frac{b - a}{n}\cdot \left(f(x_1) - f(x_0) + f(x_2) - f(x_1) + \cdots + f(x_n) - f(x_{n-1})\right)\\
    &= \frac{b - a}{n}\cdot \left(f(x_n) - f(x_0)\right)\\
    &= \frac{b - a}{n}\cdot \left(f(b) - f(a)\right)\\
\end{align*}    
Chose \(n > \frac{(b - a)(f(b) - f(a))}{\varepsilon}\).\\
Then, \(U(f, P) - L(f, P) < \varepsilon\) and so \(f\) is integrable.
\end{prf}
\begin{explanation}{}
    This theorem states that if a function is monotone, it is integrable. The proof is fairly simple, it requires us to find a partition that splits the interval evenly. We then expand the upper and lower sum which allows us to find a simple expression for the upper - lower sum. We then choose n to be larger than something which allows the expression to be less than epsilon and we are done.\\
\end{explanation}




\section{The Mean Value Theorem For Integrals}
\begin{theo}[The Mean Value Theorem For Integrals]{}
If \(f\) is continuous on \([a, b]\), then, there exists some \(C \in [a, b]\) such that \(\integral_a^b f(x) \deriv x = f(c)(b-a)\).    
\end{theo}
\newpage
\begin{prf}{}
If \(f\) is constant on \([a, b]\), the result is obvious.\\
Assume \(f\) is not constant on \([a, b]\).\\
Denote \(m  = \displaystyle\inf_{[a, b]}f(x), M = \displaystyle\sup_{[a, b]}f(x)\).\\
Since \(f\) is continuous, by the extreme value theorem, there exists points \(x_m \in [a, b]\) and \(x_M \in [a, b]\) such that \(m = f(x_m), M = f(x_M)\).\\
Without loss of generality, assume \(x_m < x_M\). Observe that 
\[m(a-b) \le \integral_a^b f\deriv x \le M(b-a)\]
\[m \le \frac{\integral_a^b f\deriv x}{(b - a)} \le M\]
Apply the intermediate value theorem on \([x_m, x_M]\). Since \(f(x_m) = m\) and \(f(x_M) = M\), there exists some \(C \in [x_m, x_M]\) such that \(f(C) = \frac{\integral_a^b f\deriv x}{(b-a)}\).\\
Then, \[\integral_a^b f\deriv x = f(C)\cdot (b - a)\]
\end{prf}
\begin{explanation}
    This theorem states the existance of a value such that a rectangle made of height c and width b - a will be the value of the definite integral between a and b.\\
We start by assuming the function is not constant as a constant function's area is a rectangle anyway.\\
We use the extreme value theorem to give us the global max and min at the supremum and infimum. We then use this to bound the integral from either side. Rearranging the inequality, we can then apply the intermediate value theorem which states that a continuous function will attain every point between an interval. Thus we know there is some value that gives us the inequality and so we are done.\\
\end{explanation}


\newpage
\section{Continuity of an Antiderivative}
\begin{theo}[Continuity of an Antiderivative]{}
If \(f\) is integrable on \([a, b]\) then the function \(F(x) = \integral_a^b f(x) \deriv x\) is continuous on \([a, b]\).   
\end{theo}

\begin{prf}{}
Suppose \(C \in [a, b]\). We will show F is continuous at C. Let M be such that \(|F(x)| \le M \quad \forall x \in [a, b]\).\\
We want to show that \(\Lim_{h\to 0} F(C+h) = F(C)\) or equivalently, \(\Lim_{h\to 0} (F(C+h) - F(C)) = 0\).\\
We begin by showing \(\lim_{h\to 0^+}(F(C+h) - F(C)) = 0\).\\
Indeed, if \(h > 0\) then, \[F(C+h) - F(C) = \integral_a^{C+h} f\deriv x - \integral_a^C f \deriv x = \integral_C^{C+h} f \deriv x\]
Since \(-M \le f \le M\) on \([a, b]\),
\[\integral_C^{C+h} f\deriv x \le M(\cancel{C}+ h - \cancel{C}) = Mh\]
Also, \(-Mh \le \integral_C^{C+h}f\deriv x\).\\
Therefore, \[|f(C+h)-f(C)| = \bigg\vert\integral_C^{C+h} f\deriv x\bigg\vert \le Mh\]
This means \(\lim_{h\to 0^+}(F(C+h) - F(C)) = 0\) and similarly, \(\lim_{h\to 0^-}(F(C+h) - F(C)) = 0\).\\
\end{prf}
\begin{explanation}{}
    This theorem states that if a function is continous, the integral of the function will be continuous.\\
The proof shows that the limit is equal to the value for any arbitrary point. We can than take the bound of the function and evaluate the definite integral to give an expression which is less that the integral. This allows us to create an inequality with absolute values which is exactly what we want when we take h to 0 as it will just give the limit we need.\\
\end{explanation}

\newpage
\section{The Fundamental Theorem of Calculus}
\begin{theo}[The Fundamental Theorem of Calculus]{}
Assume \(f: [a, b]\to \mathbb R\) is continuous.\\
Define \(F(x) = \integral_a^x f(t) \deriv t\). Then, F is differentiable on \((a, b), F'(x) = f(x)\).\\
Indeed, \(\integral_a^b f \deriv x = F(b) - F(a)\).    
\end{theo}
\begin{prf}{}
Let us compute \(F'(x)\) for \(x\in(a, b)\). We want to find \(\Lim_{h\to 0} \frac{F(x+h) - F(x)}{h}\).\\
We begin by finding \(\Lim_{h\to 0^+} \frac{F(x+h) - F(x)}{h}\).\\
Given \(h\in (0, b-x)\), we compute, \[F(x+h) - F(x) = \integral_x^{x+h}f(t)\deriv t\]
By the mean value theorem for integrals,
\[\integral_x^{x+h} f(t) \deriv t = f(c)(\cancel{x}+ h - \cancel{x}) = f(c)(h)\]
Where \(c\in [x, x+h]\).\\
Note that \(f\) is uniformly continuous on \([a, b]\). Therefore, given \(\varepsilon > 0\), there exists \(\delta > 0\) such that if \(|x-y| < \delta\) then \(|f(x) - f(y)| < \varepsilon\).\\
If \(h < \delta\) then \(|c- x| < \delta\). Therefore, \[\bigg\vert\frac{F(x+h) - F(x)}{h} - f(x)\bigg\vert = \bigg\vert\frac{f(c)h}{h}-f(x)\bigg\vert = |f(c) - f(x)| < \varepsilon\]
Thus, \(\Lim_{h\to 0^+} \frac{F(x+h) - F(x)}{h} = f(x)\).\\
Similarly, \(\Lim_{h\to 0^-} \frac{F(x+h) - F(x)}{h} = f(x)\).\\
So, \(F'(x) = f(x)\).\\
The Newton-Leibniz formula follows immediately.
\end{prf}
\newpage
\begin{explanation}{}       
    This is one of the most important theorems on this list. It links the derivative and the integral. We have that the derivative of an integral gives the function itself and we also have the Newton-Leibniz formula which allows us to take definite integrals.\\
We begin the proof by finding the derivative of the integral. Expanding this as an integral, we can use the mean value theorem to give us a simple expression of this integral.\\
Since f is continous on the closed bounded interval, we can say it is uniformly continuous. This allows us to use the definition of uniform continuity. We take the definition of the derivative and this simplifies down to exactly the inequality given in uniform continuity and this shows the limits are equal showing it is differentiable.\\
\end{explanation}


\section{Substitution Formula}
\begin{theo}[Substitution Formula]{}
If \(f, g\) are continuous on \([a, b]\) and \(g\) is differentiable on \((a, b)\) with \(g'\) continuous, then,\\
\[\integral_{g(a)}^{g(b)} f(u) \deriv u = \integral_a^b f(g(x))g'(x)\deriv x\]    
\end{theo}
\begin{prf}{}
Suppose F is an antiderivative of \(f\).\\
By the fundamental theorem of calculus, the left hand side of the equation is equal to, 
\[F(g(b)) - f(g(a))\]
Next, by the chain rule, we have, \[(F\circ g)' = (F'\circ g)\cdot g' = (f\circ g)g'\]
This means \(F\circ g\) is an anti-derivative of \((f\circ g)g'\).\\
Therefore, by the fundamental theorem of calculus, 
\begin{align*}
\integral_a^b (f\circ g)g'\deriv x &= (F\circ g)(b) - (F\circ g)(a)\\
&= F(g(b)) - F(g(a))
\end{align*}
Thus, \(\integral_{g(a)}^{g(b)} f(u) \deriv u = \integral_a^b f(g(x))g'(x)\deriv x\) holds.
\end{prf}
\begin{explanation}{}
    Most of the proofs with integration will now fall out using the fundamental theorem. This is one such proof. We say that we can take some integral and substitute a function to reverse the chain rule for derivatives.\\
The proof takes the fundamental theorem of calculus and uses it to simplify the integral expression. Then deriving the expression we use the chain rule. Now, we integrate the function we found and that will give us the right hand side.\\
\end{explanation}


\section{Integration by Parts}
\begin{theo}[Integration by Parts]{}
If \(u, v: [a, b] \to \mathbb R\) are continuous on \([a, b]\) and differentiable on \((a, b)\),
\begin{align*}
    \integral uv' \deriv x &= uv - \integral u'v \deriv x\\
    \integral_a^b uv' \deriv x &= uv\rvert_a^b - \integral_a^b u'v \deriv x 
\end{align*}    
Where \(uv\rvert_a^b = u(b)v(b) - u(a)v(a)\).\\
\end{theo}
\begin{prf}{}
We know that \(uv' = u'v + v'u\).\\
Integrating both sides, we obtain,
\begin{align*}
    \integral (uv)' \deriv x &= \integral u'v\deriv x + \integral v'u \deriv x\\
    uv &= \integral u'v\deriv x + \integral uv' \deriv x\\
    \integral u'v \deriv x &= uv - \integral v'u \deriv x
\end{align*}    
The formula for the definite integral follows from the fundamental theorem of calculus.\\
\end{prf}
\newpage
\begin{explanation}{}
    This theorem is really a justification of a formula used to integrate the product rule. It takes the product of two functions. Integrating both sides of the product rule gives us exactly the integration by parts formulua.
\end{explanation}


\section{Integral Test}
\begin{theo}[Integral Test]{}
Assume \(f\) is a non negative, non increasing continuous function on the interval \([1, \infty)\).\\
Then, \(\displaystyle\int_{1}^\infty f \deriv x\) and \(\displaystyle\sum_{n=1}^\infty f(n)\) converge or diverge together.\\     
\end{theo}
\newpage
\begin{prf}{}
Consider the interval \([1, n+1]\). The set \(P = \{1, 2, 3, 4, \cdots , n+1\}\) is a partition of \([1, n+1]\). Since \(f\) is non increasing,
\[\displaystyle\inf_{[x_{i-1}, x_i]} f = f(x_i), \sup_{[x_{i-1},x_i]} f = f(x_{i-1})\]
Consequently,
\begin{align*}
    U(f,P) &= \displaystyle\sum_{i=1}^n \sup_{[x_{i-1}, x_i]} f(x_i - x_{i-1})\\
    &= \displaystyle\sum_{i=1}^n f(x_{i-1})(i+1-i)\\
    &= \displaystyle\sum_{i=1}^n f(x_{i-1})\\
    &= \displaystyle\sum_{i=1}^n f(i)\\
\end{align*}
Also, \(L(f,P) = \displaystyle\sum_{i=1}^n f(i+1)\).\\
If \(\displaystyle\int_1^\infty f \deriv x\) converges, then, \(L(f, P) = \displaystyle\sum_{i=1}^n f(i+1) = \displaystyle\sum_{i=2}^{n+1}f(i)\) converges as \(n\to\infty\).\\
Thus, \(\displaystyle\sum_{i=1}^\infty f(i)\) converges.\\
If \(\displaystyle\int_1^\infty f \deriv x\) diverges then \(U(f,P) = \displaystyle\sum_{i=1}^n f(i)\) diverges as \(n\to\infty\)\\
Which means, \(\displaystyle\sum_{i=1}^\infty f(i)\) diverges.  
\end{prf}\newpage
\begin{explanation}{}
    We are given a function that is continous, non-increasing and non negative function. This very specific function is integrable and the theorem says that the integral of this function and the sum of the function will converge and diverge together. Obviously this is accurate as the sum and integral are very similar with these assumptions.\\
The proof creates a partition on an interval. We then know that we can expand the upper and lower sums into finite series form.  Rearranging this, we have a telescoping series and can then simplify the expression greatly. Similarly with the lower sum. We then take it case by case. If the integral converges, the upper sum is the lower sum which both converge. The upper sum is exactly the sum we predict will converge when n is taken to infinity.\\
Now, if the integral diverges, the upper sum will diverge as n is taken to infinity. This obviously means the infinite series diverges.\\
\end{explanation}


\section{Limit Comparison Test}
\begin{theo}[Limit Comparison Test]{}
Suppose \(a_n, b_n > 0\) for all \(n \in \mathbb N\).\\
If \(\Lim_{n\to\infty}\frac{a_n}{b_n} = c > 0\) (not \(\infty\))\\
then, \(\displaystyle\sum_{n=1}^\infty a_n\) and \(\displaystyle\sum_{n=1}^\infty b_n\) converge and diverge simultaneously.    
\end{theo}
\begin{prf}{}
Assume \(\displaystyle\sum_{n=1}^\infty b_n\) converges. Since \(\Lim_{n\to\infty}\frac{a_n}{b_n} = c\), there exists some \(N \in \mathbb N\) such that if \(n \ge N\),  \(\frac{a_n}{b_n} < 2c\) and \(a_n < 2cb_n\).\\
Now, \(\displaystyle\sum_{n=1}^\infty 2cb_n = 2c\displaystyle\sum_{n=1}^\infty b_n\) converges.\\
By comparison, since \(\displaystyle\sum_{n=N}^\infty a_n \le 2c\displaystyle\sum_{n=N}^\infty b_n \le 2c\displaystyle\sum_{n=1}^\infty b_n\), the series \(\displaystyle\sum_{n=N}^\infty a_n\). Hence, \(\displaystyle\sum_{n=1}^\infty a_n\) converges.\\
Next, suppose \(\displaystyle\sum_{n=1}^\infty a_n\) converges. Then, \(\Lim_{n\to\infty}\frac{b_n}{a_n} = \frac{1}{c} > 0\) and \(\displaystyle\sum_{n=1}^\infty b_n\) converges by the previous argument.
\end{prf}
\newpage
\begin{explanation}{}
    This theorem tests infinite series convergence and divergence by testing if the ratio of two positive sequences converges. If so, then the two series will converge or diverge together. The proof assumes that initially, one series converges. Since the ratio converges, we can use the definition of the limit. This gives us that the other sequence is less than a scalar multiple of the first sequence.\\
Using the comparison theorem which states that if $0 \le a_n \le b_n$ and $\displaystyle\sum_{n=1}^\infty b_n$ converges, then $\displaystyle\sum_{n=1}^\infty a_n$ converges. Indeed, we know that the first sequence is larger than the second and that the first converges. This gives us that the second converges.\\
Next we look at if the second series converges. Then, the previous argument follows and we get that they converge together.\\
\end{explanation}


\section{Ratio Test}
\begin{theo}[Ratio Test]{}
Assume \(a_n > 0\) for all \(n \in \mathbb N\).\\
Then, 
\begin{enumerate}
    \item If \(\displaystyle\limsup_{n\to\infty}\frac{a_{n+1}}{a_n} = r < 1\)\\
Then, \(\displaystyle\sum_{n=1}^\infty a_n\) converges.
    \item If \(\displaystyle\liminf_{n\to\infty} \frac{a_{n+1}}{a_n} > 1\)\\
Then, \(\displaystyle\sum_{n=1}^\infty a_n\) diverges.
\end{enumerate}
\end{theo}
\begin{prf}{}
We assume that \(\Lim_{n\to\infty} \frac{a_{n+1}}{a_n} = r \in \mathbb R\).\\
\begin{enumerate}
    \item Fixing \(S\) such that \(r < S < 1\).\\
    Because \(\Lim_{n\to\infty} \frac{a_{n+1}}{a_n} = r < S\), there exists some \(N \in \mathbb N\) such that if \(n \ge N\) then \(\frac{a_{n+1}}{a_n} < S\).\\
    Then, \(\frac{a_{N+1}}{a_N} < S\) and \(a_{N+1} < S\cdot a_N\).\\
    Also, \(\frac{a_{N+2}}{a_{N+1}} < S\) then \(a_{N+2} < S\cdot a_{N+1} < S^2 a_{N}\).\\
    Next, \(\frac{a_{N+3}}{a_{N+2}} < S\) then \(a_{N+3} < S\cdot a_{N+2} < S^3 a_{N}\).\\
    Continuing like this, we find \(a_{N+k} < S^k \cdot a_N\). Thus, \[\displaystyle\sum_{k=N}^\infty a_n \le \displaystyle\sum_{k=0}^\infty S^k\cdot a_N = a_N\displaystyle\sum_{k=0}^\infty S^k = a_N\frac{1}{1-S}\]
    Thus, \(\displaystyle\sum_{k=N}^\infty a_N\) converges. Hence, \(\displaystyle\sum_{n=0}^\infty a_n\) converges.
    \item Assuming \(\Lim_{n\to\infty}\frac{a_{n+1}}{a_{n}} = r > 1\) then there exists some \(N \in \mathbb N\) such that if \(n \ge N\) then \(\frac{a_n+1}{a_n} > 1\) then \(\frac{a_{N+1}}{a_N} > 1\) so, \(a_{N+1} > a_N\). Also, \(\frac{a_{N+2}}{a_N+1} > 1\) so \(a_{N+2} > a_{N+1} > a_{N}\) and so on.\\
    Thus, \(a_{N+k} > a_{N}\) for all \(k \in \mathbb N\) which means \(a_n\) cannot go to \(0\) as \(n\to\infty\).
\end{enumerate}  
\end{prf}
\begin{explanation}{}
    This theorem is very strong at proving series convergence or divergence.\\
We say that taking the limit of the ratio of two consecutive terms. If the limit is less than 1, the series will converge. If it is greater than 1, the series will diverge. For the first statement, we assume some value larger than the limit. We then say that by definition, the ratio will be less than that value. Rearranging this, we can see than the larger term of the sequence is less than the value multiplied by the smaller term. We then apply this idea multiple times, creating an inequality for the larger term of the sequence. We then take the infinite sum of this inequality and see that the series past a certain point converges which implies the series from 1 to infinity converges.\\
For the second statement, we know the ratio converges to something less than 1. Applying the definition of the limit, we find that the $a_{n+1} > a_n$. This follows for all n and so we have some sequence that cannot converge to 0 which implies divergence. 
\end{explanation}


\section{The Leibniz Test}
\begin{theo}[The Leibniz Test]{}
Assume \(\left(a_n\right)_{n=1}^\infty\) is a sequence such that \(a_n \ge 0\), \(a_n \ge a_{n+1}\) for all \(n\in \mathbb N\) and \(\Lim_{n\to\infty} a_n = 0\),\\
Then, the series \(\displaystyle\sum_{n=1}^\infty(-1)^{n+1}a_n\) converges.\\   
\end{theo}
\newpage
\begin{prf}{}
Let \(S_n  = \displaystyle\sum_{k=1}^n(-1)^{k+1}a_k\).\\
Observe that \(S_1 \ge S_3 \ge S_5 \ge \cdots\)\\
Indeed, 
\begin{align*}
    S_{2n+3} &= S_{2n+1} - a_{2n + 2} + a_{2n+3}\\
    &= S_{2n+1} + \left(a_2n+3 - a_2n+2\right)\\
    &\le S_{2n+1}
\end{align*}  
Also, \[S_{2n+2} = S_{2n} + \left(a_{2n+1} - a_{2n+2}\right) \ge S_{2n} \]
And, \[S_2 \le S_4 \le S_6 \le \cdots\]
Moreover, \(S_{2n} \le S_{2n+1}\) since \(S_{2n+1} = S_{2n} + a_{2n+1} \ge S_{2n}\).\\
We conclude \(S_{2n} \le S_{2n+1} \le S_1\) and \(S-{2n+1} \ge S_{2n} \ge S_{2}\).\\
Thus, the sequences \(\left(S_{2n}\right)_{n=1}^\infty\) and \(\left(S_{2n+1}\right)_{n=1}^\infty\) are bounded.\\
Hence, they must converge. Thus, we can say, 
\[\Lim_{n\to\infty}S_{2n} = \alpha, \quad \Lim_{n\to\infty}S_{2n+1} = \beta\]
Now, 
\begin{align*}
\beta -\alpha &= \Lim_{n\to\infty}S_{2n+1} - \Lim_{n\to\infty}S_{2n}\\
&= \Lim_{n\to\infty}\left(S_{2n+1} - S_{2n}\right)\\
&= \Lim_{n\to\infty}a_{2n+1}\\
&= 0
\end{align*}
Thus, \(\beta = \alpha\) and \(\Lim_{n\to\infty}S_n = \alpha = \beta\) and \(\displaystyle\sum_{n=1}^\infty(-1)^{n+1}a_n\) converges.
\end{prf}
\newpage
\begin{explanation}{}
    This theorem says that if we have a monotone decreasing sequence and the limit of the sequence is 0, then a series with alternating sign multiplied by the sequence will converge.\\
The proof constructs some sequence that is the sum from 1 to n of an alternating sign sequence. We then know that the odd values of the sum are going to be smaller than the initial value. From that, we can generalise an expression that the created sequence is decreasing. Now, we look at the even terms. We can write these as the sum and difference of previous terms which we know is less than a smaller even term. Thus, the even terms increase in size. So, we know that The evens are less than the odd terms. This gives us that the odd terms and even terms are bounded so they must converge. Now we must prove they converge to the same limit. With a small amount of algebra this is obvious and so we have proven the Leibniz test.\\
\end{explanation}


\section{An Absolutely Convergent Series implies convergence}
\begin{theo}[An Absolutely Convergent Series implies convergence]{}
If \(\displaystyle\sum_{n=1}^\infty a_n\) converges absolutely, then it converges.   
\end{theo}
\begin{prf}{}
Use the cauchy criterion.\\
Fixing \(\varepsilon > 0\). We must show that there exists some \(N \in \mathbb N\) such that if \(p > q \ge N\) then \[\left|\displaystyle\sum_{n = q}^p a_n\right| < \varepsilon\]\\
Since \(\displaystyle\sum_{n=1}^\infty |a_n|\) converges, there exists some \(N \in \mathbb N\) such that \(\left|\displaystyle\sum_{n = q}^p |a_n|\right| < \varepsilon\)\\
By the triangle inequality, \[\left|\displaystyle\sum_{n = q}^p a_n\right| \le \displaystyle\sum_{n=1}^\infty |a_n| = \left|\displaystyle\sum_{n = q}^p |a_n|\right| < \varepsilon \]
\end{prf}
\newpage
\begin{explanation}{}
    This theorem may seem obvious but it is an important result. If a series is absolutely convergent then it converges.\\
The proof uses the cauchy criterion which is included at the bottom of this document. The triangle inequality solves this proof entirely.\\
\end{explanation}


\section{Maclaurin Series Examples}
\begin{theo}[Maclaurin Series Examples]{}
The formulas and derivations for the functions, 
\begin{enumerate}
    \item \(f(x) = e^x\)
    \item \(f(x) = \sin(x)\)
    \item \(f(x) = \cos(x)\)
    \item \(f(x) = \frac{1}{1-x}\)
\end{enumerate}    
\end{theo}
\begin{prf}{}
    \begin{enumerate}
        \item Let \(f(x) = e^x\).\\
        Let us write down the Maclaurin Series.\\
        Clearly, \(f^{(n)}(0) = \frac{\deriv^n}{\deriv x^n}e^x\bigg\vert_{x=0} = e^0 = 1\)\\
        Therefore, the series is \(\displaystyle\sum_{n=0}^\infty \frac{x^n}{n!}\).\\
        This converges absolutely for all \(x\in \mathbb R\) by the ratio test.
        \item Let \(f(x) = \sin(x)\).\\
        \begin{align*}
            f(x) &= \sin(x)\\
            f'(x) &= \cos(x)\\
            f''(x) &= - \sin(x)\\
            f'''(x) &= - \cos(x)\\
            f''''(x) &= \sin(x)
        \end{align*}
        \[f^{4k} (x) = \sin(x) \qquad f^{4k+1}(x) = \cos(x)\]
        For all \(k \in \mathbb{N}\)
        \begin{align*}
            \displaystyle\sum^\infty_{n=0} \frac{f^{(n)}(0)}{n!} \cdot x^n &= \frac{\sin(0)}{0!} \cdot x^0 + \frac{\cos(0)}{1!} \cdot x^1 + \frac{- \sin(0)}{2!} \cdot x^2 + \frac{- \cos(0)}{3!}\cdot x^3 + \dots\\
            &= x - \frac{1}{3!} \cdot x^3 + \frac{1}{5!} \cdot x^5\\
            &= \displaystyle\sum^\infty_{n=0} (-1)^n \cdot \frac{x^{2n+1}}{(2n+1)!}
        \end{align*}
        \underline{Note:} the series converges absolutely by the ratio test for all \(x \in \mathbb{R}\)
        \[\limsup_{n \to \infty} \frac{\left(-1\right)^{n+1}\cdot \left(\frac{x^{2n+2}}{\left(2n+2\right)!}\right)}{\left(-1\right)^n\cdot \left(\frac{x^{2n+1}}{\left(2n+1\right)!}\right)} = \limsup_{n \to \infty} \frac{-x}{2n+2} = 0 < 1\]       
        \item Let \(f(x) = \cos(x)\).\\
        \begin{align*}
            f(x) &= \cos(x)\\
            f'(x) &= - \sin(x)\\
            f''(x) &= - \cos(x)\\
            f'''(x) &= \sin(x)\\
            f''''(x) &= \cos(x)
        \end{align*}
        \begin{align*}
            \displaystyle\sum^\infty_{n=0} \frac{f^{(n)}(0)}{n!} \cdot x^n &= \frac{\cos(0)}{0!} \cdot x^0 + \frac{- \sin(0)}{1!} \cdot x^1 + \frac{- \cos(0)}{2!} \cdot x^2 + \frac{\sin(0)}{3!}\cdot x^3 + \dots\\
            &= 1 - \frac{1}{2!} \cdot x^2 + \frac{1}{4!} \cdot x^4 - \frac{1}{6!} \cdot x^6 + \dots\\
            &= \displaystyle\sum^\infty_{n=0} (-1)^{n} \cdot \frac{x^{2n}}{2n!}
        \end{align*}
        For all \(x \in \mathbb{R}\)
        \begin{align*}
            f(x) &= \frac{1}{1-x}\\
            f'(x) &= \frac{1}{(1-x)^2}\\
            f''(x) &= \frac{2}{(1-x)^3}\\
            f'''(x) &= \frac{6}{(1-x)^4}\\
            f''''(x) &= \frac{24}{(1-x)^5}
        \end{align*}
        \begin{align*}
            \displaystyle\sum^\infty_{n=0} \frac{f^{(n)}(0)}{n!} \cdot x^n &= \frac{1}{(1-0)(0!)} \cdot x^0 + \frac{1}{(1-0)^2(1!)} \cdot x^1 + \frac{2}{(1-0)^3(2!)} \cdot x^2\\
            &+ \frac{6}{(1-0)^4(3!)} \cdot x^3 + \frac{24}{(1-0)^5(4!)} \cdot x^4\\
            &= \displaystyle\sum^\infty_{n=0} \frac{n!}{n!} \cdot x^n\\
            &= \displaystyle\sum^\infty_{n=0} x^n
        \end{align*}
        This is a geometric series. It converges if \(x \in (-1,1)\) and diverges otherwise.
    \end{enumerate} 
\end{prf}
\begin{explanation}{}
    This isnt really a theorem, it is a collection of examples of common Maclaurin series.\\
A Maclaurin series is an approximation of a function at values close to 0.\\
The derivation of each example has an infinite sum of the ratio of the nth derivative and n factorial multiplied by x to the power of n. So, for each function we need to find the nth derivative and we are done.\\
\end{explanation}


\section{Inhomogenous Linear System Solutions}
\begin{proposition}[Inhomogenous Linear System Solutions]{}
If \(p\) is a vector such that \(Ap = b\), then,
\[\{x\in \mathbb R^n | Ax = b\} = \{y + p | y\in \rm{NS}(A)\}\]
\end{proposition}
\begin{prf}{}
Observe that \(A(x+y) = Ax + Ay\) and \(A(\lambda x) = \lambda Ax\) for all \(x, y \in \mathbb R^n\), \(\lambda \in \mathbb R\).\\
Assume \(y \in \rm{NS}(A)\). Let us prove that \(A(y + p) = b\).\\
Indeed, \(A(y+p) = Ay + Ap = 0 + b = b\).\\
Thus,\[\{x\in \mathbb R^n | Ax = b\} \supset \{y + p | y\in \rm{NS}(A)\}\]
Now, assume \(Ax = b\). Clearly, \(x = p + (x - p)\). \\
The vector \(y = x - p\) is in \(\rm{NS}(A)\) because \[A(x - p) = Ax - Ap = b - b = 0\]
Thus, \[\{x\in \mathbb R^n | Ax = b\} \subset \{y + p | y\in \rm{NS}(A)\}\]
\end{prf}
\begin{explanation}{}
    This proposition states that if we have some vector multiplied by a matrix, then the vector is given by some x - the null space of the matrix.\\
All of the solutions to Ax = b (including the one you found, p) are the same as 1 solution to Ax = 0 and adding p.\\
We start the proof with some properties assumed on the matrix and defining variables. Then, we need to prove $A(y+p) = 0$ as this will ensure $x = y + p$. This is quite obvious. Indeed, we also need to show that $x = p + y$ which is also quite obvious. Then we are done.\\
\end{explanation}


\section{Invertibility of a Matrix and the Null Space}
\begin{theo}[Invertibility of a Matrix and the Null Space]{}
The matrix A is invertible if and only if \(\rm{NS}(A) = \{0\}\).
\end{theo}
\begin{prf}{}
\(\left(\implies\right)\)\\
Assume A is invertible. We know \(0 \in \rm{NS}(A)\). We want to prove that if \(Ax = 0, x = 0\) then,
\begin{align*}
x &= (A^{-1}A)x\\
&= A^{-1}(Ax) = A^{-1}0 = 0
\end{align*}
\(\left(\impliedby\right)\)\\
We claim that if \(\rm{NS}(A) = 0\) then A has a right inverse.\\
Proof of the claim\\
If \(\rm{NS}(A = 0\) then the system \(Ax = 0\) has a unique solution.\\
By an earlier proposition, the system \(Ax = b\) has a unique solution for every \(b \in \mathbb R^n\).\\
Take,
\[
b^1 = \begin{pmatrix}
    1 \\ 0 \\ \vdots\\ 0
\end{pmatrix},
\qquad b^2 = \begin{pmatrix}
    0 \\ 1 \\0\\ \vdots\\ 0
\end{pmatrix},
\qquad \cdots 
\]
Solving \(Ax = b^i\) for \(i = 1, 2, \cdots, n\), we obtain an array of vectors,
\[x^i = 
\begin{pmatrix}
    x_{1i} \\ x_{2i}\\ \vdots\\ x_{ni}
\end{pmatrix}
\]
We construct a matrix, 
\[C = \begin{pmatrix}
    x_{11} & \cdots &x_{1n}\\
    \vdots & \ddots & \vdots\\
    x_{n1} & \cdots & x_{nn}
\end{pmatrix}\]
Clearly, \(AC = I_n\).\\
Thus, C is the right inverse of A.\\
This proves the claim.\\
Now, let us show that A has a left inverse. Note that C has a left inverse, namely, A.\\
By the same argument as in \(\left(\implies\right)\).\\
This implies \(\rm{NS}(C) = \{0\}\). By the claim, C has a right inverse.\\
This inverse is equal to A (proven earlier).\\
Thus, \(AC = CA = I_n\).\\
Thus, C is the inverse of A.
\end{prf}
\begin{explanation}{}
    This theorem looks at the nullspace of a matrix and concludes that if is zero then the matrix is invertible. Indeed, if the matrix is invertible the null space is zero.\\
The forward direction follows directly from the definition of invertibility.\\
The reverse direction is a bit more involved.\\
We must prove that if the nullspace is zero then the matrix is invertible.\\
We start by showing it has a right inverse. We construct a matrix using the unique solutions to the system $Ax = b$. This is clearly the inverse of A on the right as we have defined the matrix of b to be the identity matrix.\\
Now, we show that A has a left inverse. Obviously, A is the left inverse of C. Thus, the nullspace of C is zero and so we have shown that C has a right inverse. This is just A. This proves that A is invertible and we are done.\\
\end{explanation}


\newpage
\section{Linearity of the Determinant}
\begin{theo}[Linearity of the Determinant]{}
Suppose \(u, v, a_1, \cdots, a_n\) are vectors in \(\mathbb R^n\). Consider the matrices,
\[
A = \begin{pmatrix}
    a_1\\ a_2 \\ \vdots \\ a_{r-1} \\ u + \lambda v \\ a_{r+1} \\ \vdots \\ a_n
\end{pmatrix}  
\qquad
B = \begin{pmatrix}
    a_1\\ a_2 \\ \vdots \\ a_{r-1} \\ u \\ a_{r+1} \\ \vdots \\ a_n
\end{pmatrix} 
\qquad
C = \begin{pmatrix}
    a_1\\ a_2 \\ \vdots \\ a_{r-1} \\ v \\ a_{r+1} \\ \vdots \\ a_n
\end{pmatrix} 
\]
Then, \(\det A = \det B + \lambda \det C\)
\end{theo}
\begin{prf}{}
We will argue with induction.\\
The result is obvious if \(n = 1\). Then, \(A = (u + \lambda v), B = u,  C = v\) for some \(u, v \in \mathbb R\).\\
Then, \[\det A = u + \lambda v = \det B + \lambda \det C\]
Assume the result holds for \((n-1)\times (n-1)\) matrices. Let us prove the result for \(A, B, C\) being \(n\times n\) matrices.\\
\underline{Case 1:}\\
Assume \(r = 1\).\\
In this case, 
\begin{align*}
\det A &= \displaystyle\sum_{j = 1}^n (-1)^{1+j} (u_j + \lambda v_j)\det\tilde{A_{1j}}\\
&= \displaystyle\sum_{j=1}^n (-1)^{1+j}u_j \det \tilde{A}_{1j} + \lambda\displaystyle\sum_{j=1}^n(-1)^{1+j}v_j \det \tilde{A_{1j}}\\
\end{align*}
Where \(u = (u_1, u_2, \cdots u_n)\qquad v = (v_1, v_2, \cdots, v_n)\).\\
Now, since \(r = 1\), \(\tilde{A_{1j}} = \tilde{B_{1j}} = \tilde{C_{1j}}\).\\
Therefore, 
\begin{align*}
\det A &= \displaystyle\sum_{j = 1}^n (-1)^{1+j} u_j \det \tilde{B_{1j}} + \lambda\displaystyle\sum_{j = 1}^n (-1)^{1+j} v_j \det \tilde{C_{1j}}\\
&= \det B + \lambda\det C
\end{align*}
\underline{Case 2:}\\
Assume \(r > 1\).\\
In this case, the first rows of \(A, B\) and \(C\) are the same. In fact, they are \(a_1\).\\
Now, \[\det A = \displaystyle\sum_{j=1}^n (-1)^{1+j}A_{1j}\det\tilde{A_{1j}}\]
The matrix \(\tilde{A_{ij}}\) is \((n-1)\times (n-1)\). By the induction hypothesis, \[\det \tilde{A_{1j}} = \det \tilde{B_{1j}} + \lambda\det\tilde{C_{1j}}\]
Therefore,
\begin{align*}
    \det A &= \displaystyle\sum_{j=1}^n (-1)^{j+1} A_{1j}\det\left(\tilde{B_{1j}}+ \tilde{C_{1j}}\right)\\
    &= \displaystyle\sum_{j=1}^n (-1)^{j+1} A_{1j}\det\tilde{B_{1j}} + \lambda\displaystyle\sum_{j=1}^n (-1)^{j+1} A_{1j}\det\tilde{C_{1j}}\\
    &= \displaystyle\sum_{j=1}^n (-1)^{j+1} B_{1j}\det\tilde{B_{1j}} + \lambda\displaystyle\sum_{j=1}^n (-1)^{j+1} C_{1j}\det\tilde{C_{1j}}\\
    &= \det B + \lambda \det C
\end{align*}
\end{prf}
\begin{explanation}{}
    This theorem is very useful in calculating determinants. It says that if we have a matrix that is the same as another matrix with a row containing some augmentation of the other row, we can take the determinant equal to the det of the similar matrix and the scalar multiple of a similar matrix again.\\
We use induction for this proof. The base case of a 1x1 matrix is obvious since the determinant is just the element in the matrix which holds under addition and multiplication.\\
We assume it holds for $(n-1)x(n-1)$ matrices and now prove it holds for nxn matrices.\\
We write the determinant of A as the expanded series formula. We can then split this into 2 series. Whih is obviously the result we want. This holds when the row in question is the first row.\\
We now look at when the row in question is anywhere. The definition of the determinant includes a matrix which is $(n-1)x(n-1)$ by the assumption, the theorem holds here. We can rewrite this then as the determinant of the smaller B and C. Expanding this again we then get the result we need.\\
\end{explanation}

\newpage
\section{Invertibility and the Determinant}
\begin{theo}[Invertibility and the Determinant]{}
    A matrix A is invertible if and only if \(\det A \neq 0\)
\end{theo}
\begin{prf}{}
Suppose A is some square matrix.
\(\left(\implies\right)\)\\
Since \(AA^{-1} = I_n\), taking the determinant of this,\\
\[\det A \det A^{-1} = 1\]
This clearly shows \(\det A \neq 0\).\\
\(\left(\impliedby \right)\)\\
Cosider the matrix \[G = \frac1{\det A}\left(\left(C_{ij}\right)_{i,j = 1}^n\right)^T\]
Where \(C_{ij} = (-1)^{i+j} \det \tilde{A_ij}\).\\
We claim that \[GA = AG = I_n\]
Indeed, given \(k = 1, \cdots, n\), we find,
\begin{align*}
    \left(AG\right)_{KK} &= \displaystyle\sum_{i=1}^n A_{ki} G_{ki} \\
    &= \displaystyle\sum_{i=1}^n A_{ki}\frac{1}{\det A}(-1)^{i+k} \det \tilde{A_{ki}}\\
    &= \frac1{\det A}\displaystyle\sum_{i=1}^n A_{ki}(-1)^{i+k}\det \tilde{A_{ki}}\\
    &= \frac{1}{\det A} \det A = 1
\end{align*}
Thus, \(AG\) has a diagonal of 1s.\\
If \(K \neq L\) then,\\
\begin{align*}
    \left(AG\right)_{KL} &= \displaystyle\sum_{i=1}^n A_{Ki}G_{iL}\\
    &= \frac{1}{\det A} \displaystyle\sum_{i=1}^n A_{Ki}(-1)^{i+L} \det \tilde{A_{Li}}
\end{align*}
The sum is the determinant of the matrix
\[\begin{pmatrix}
    A_{11} & \cdots & A_{1N}\\
    \vdots & & \vdots\\
    A_{K1} & \cdots & A_{KN}\\
    \vdots & \ddots & \vdots\\
    A_{K1} & \cdots & A_{KN}\\
    \vdots & \ddots & \vdots\\
    A_{N1} & \cdots & A_{NN}
\end{pmatrix}\]
However, this matrix has two identical rows so its determinant is equal to 0.\\
This means that \(AG\) has zeroes off the diagonal.\\
\[AG = GA = I_n\]
Thus, \(A\) is invertible.
\end{prf}
\begin{explanation}{}
    This theorem says that a matrix is invertible iff the determinant is not 0. That is, we can find an inverse to any square matrix that has a determinant not 0.\\
The forward direction is very simple.\\
The reverse direction considers a new matrix that is defined as the reciprocal of the determinant multiplied by the transpose of another matrix.\\
We want to show that this matrix is the inverse of A.\\
This is done by expanding the multiplication of $(AG)_{KK} $into series form. We can take the reciprocal of det A out as it is just a constant. This then gives us the formula for the determinant of A. This means that any point of KK in AG is 1. Thus, the diagonal is 1.\\
We now look at everyone off the diagonal. Expanding this into series form, we can again take out the reciprocal of det A. We can change the value of L to K as it is just a determinant at a different row. Indeed, when taking the determinant at row L, we will take it out. Thus, we can interchange the variables. This gives us a matrix with 2 identical rows which obivously has determinant of 0. So, we have shown it has zeros off the diagonal. This is just the identity matrix and we are done.\\
\end{explanation}


\begin{center}
    \textmd{Authors note}
\end{center}
\vspace{10mm}
These following proofs are not on the final list but are useful or used throughout as assumed knowledge.\\
Thanks.
\newpage
    \section{Cauchy Criterion}
    \begin{theo}[Cauchy Criterion]{}
        The series
        \[\displaystyle\sum^\infty_{n=1} a_n\]
        converges if and only if for all \(\varepsilon >0\) there exists \(N \in\mathbb{N}\) such that if \(q> p \geq N\) then
        \[\bigg\vert\displaystyle\sum^q_{k=p+1} a_k\bigg\vert < \varepsilon\]
    \end{theo}
    \begin{prf}
        Denote the partial sum as follows,
        \[s_k = \displaystyle\sum^k_{n=1} a_n\]
        \(\displaystyle\sum^\infty_{n=1} a_n\) converges if and only if \((s_n)^\infty_{n=1}\) converges if and only if \((s_n)^\infty_{n=1}\) is Cauchy. This means for all \(\varepsilon > 0\) there exists \(N \in \mathbb{N}\) such that if \(p,q \geq N\)
        \[\vert s_q - s_p\vert < \varepsilon\]
        Now,
        \[\vert s_q - s_p\vert = \bigg\vert\displaystyle\sum^q_{n=1} a_n - \sum^p_{n=1} a_n\bigg\vert = \bigg\vert\displaystyle\sum^q_{n=p+1} a_n\bigg\vert < \varepsilon\]
    \end{prf}
\end{document}


